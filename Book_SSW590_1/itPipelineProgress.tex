\chapter{Pipeline Progress \& Build Automation \\
\small{\textit{-- Build System Visibility and Automation}}
\index{Pipeline Progress}
\index{Chapter!Pipeline Progress}
\label{Chapter::PipelineProgress}

\section{Overview}
\label{Section::PipelineProgress::Overview}

This chapter describes the implementation of a comprehensive CI/CD pipeline that provides real-time visibility into the build and compilation process. The pipeline automates the compilation of LaTeX documentation, packaging of project source code, and creation of deployment artifacts.

\section{Pipeline Architecture}
\label{Section::PipelineProgress::Architecture}

The pipeline consists of five main stages that execute sequentially, each with progress indicators and status reporting:

\begin{enumerate}
    \item \textbf{Dependency Verification} - Validates that all required tools (LaTeX, build tools) are installed
    \item \textbf{Build Directory Setup} - Prepares the output directory structure for build artifacts
    \item \textbf{LaTeX Compilation} - Compiles the LaTeX documentation with multiple passes
    \item \textbf{Project File Preparation} - Stages project source code for packaging
    \item \textbf{Archive Creation} - Packages project files into a distributable ZIP archive
\end{enumerate}

\section{LaTeX Compilation Process}
\label{Section::PipelineProgress::LaTexCompilation}

The LaTeX compilation follows the standard multi-pass approach required for proper bibliography and cross-reference resolution:

\subsection{Compilation Passes}
\label{Subsection::PipelineProgress::CompilationPasses}

\begin{enumerate}
    \item \textbf{Pass 1} - Initial pdflatex run to generate intermediate files
    \item \textbf{Bibliography Processing} - bibtex processes citations and generates bibliography
    \item \textbf{Pass 2} - Second pdflatex run to resolve bibliography references
    \item \textbf{Pass 3 (Final)} - Third pdflatex run to finalize cross-references and page numbers
\end{enumerate}

Each pass is monitored and reported to the user with timestamps and status indicators.

\subsection{Error Handling}
\label{Subsection::PipelineProgress::ErrorHandling}

The pipeline gracefully handles compilation errors by:
\begin{itemize}
    \item Capturing pdflatex output and logging warnings
    \item Continuing with subsequent stages even if LaTeX compilation fails
    \item Providing detailed error messages to guide troubleshooting
    \item Archiving the output directory regardless of compilation status
\end{itemize}

\section{Progress Reporting}
\label{Section::PipelineProgress::ProgressReporting}

\subsection{Real-Time Progress Display}
\label{Subsection::PipelineProgress::RealtimeDisplay}

The pipeline implementation provides visual progress indicators using color-coded output:

\begin{tcolorbox}[colback=gray!10, colframe=gray!50, title=Progress Output Example]
\small
\texttt{[14:23:45] Step 3/5 ... Compiling LaTeX Report}\\
\texttt{  Running pdflatex (pass 1)...}\\
\texttt{  Running bibtex for bibliography...}\\
\texttt{  Running pdflatex (pass 2)...}\\
\texttt{  Running pdflatex (pass 3 - final)...}\\
\texttt{[14:23:52] LaTeX Compilation ... Successfully created main.pdf}\\
\end{tcolorbox}

\subsection{Status Indicators}
\label{Subsection::PipelineProgress::StatusIndicators}

Output uses color coding for visual clarity:
\begin{itemize}
    \item \textbf{Green} - Successful operations
    \item \textbf{Red} - Errors or failures
    \item \textbf{Yellow} - Warnings or skipped steps
    \item \textbf{Cyan} - Information messages
    \item \textbf{Magenta} - Section headers and highlights
\end{itemize}

\section{Project Code Packaging}
\label{Section::PipelineProgress::ProjectPackaging}

\subsection{File Collection}
\label{Subsection::PipelineProgress::FileCollection}

The pipeline collects the following project files for packaging:
\begin{itemize}
    \item \texttt{demo.js} - Main application code
    \item \texttt{index.html} - Web interface markup
    \item \texttt{style.css} - Application styling
    \item \texttt{p5.js} - p5.js library
    \item \texttt{p5.sound.min.js} - Audio library for p5.js
    \item \texttt{README.md} - Project documentation
\end{itemize}

\subsection{Archive Generation}
\label{Subsection::PipelineProgress::ArchiveGeneration}

Project files are archived using standard ZIP compression with optimal compression level. The archive includes:
\begin{itemize}
    \item All source code and assets
    \item Directory structure preservation
    \item File metadata and timestamps
    \item Optimal compression for distribution
\end{itemize}

\section{Implementation Details}
\label{Section::PipelineProgress::Implementation}

\subsection{PowerShell Implementation}
\label{Subsection::PipelineProgress::PowerShell}

For Windows environments, a PowerShell script (\texttt{build-pipeline.ps1}) provides native integration:

\begin{minted}{powershell}
# Color-coded progress output
Write-ProgressStep "LaTeX Compilation" "Successfully created main.pdf" "Success"

# Multi-pass LaTeX compilation
pdflatex -interaction=nonstopmode -output-directory=$OutputDir $TexFile | Out-Null
bibtex $ReportName | Out-Null
pdflatex -interaction=nonstopmode -output-directory=$OutputDir $TexFile | Out-Null

# Archive creation using .NET compression
[System.IO.Compression.ZipFile]::CreateFromDirectory(...)
\end{minted}

\subsection{Python Implementation}
\label{Subsection::PipelineProgress::Python}

For cross-platform compatibility, a Python implementation (\texttt{build-pipeline.py}) is provided:

\begin{minted}{python}
class PipelineBuilder:
    def compile_latex(self, output_path):
        """Compile LaTeX report with multi-pass approach"""
        self.run_command(
            f"pdflatex -interaction=nonstopmode -output-directory={self.output_dir} {tex_file}",
            "pdflatex pass 1"
        )
        # Additional passes and bibtex processing...
        
    def create_archive(self, output_path, project_dir):
        """Create zip archive with progress reporting"""
        shutil.make_archive(str(output_path / "project"), 'zip', ...)
\end{minted}

\section{Pipeline Execution}
\label{Section::PipelineProgress::Execution}

\subsection{Triggering the Pipeline}
\label{Subsection::PipelineProgress::Triggering}

The pipeline can be triggered through multiple mechanisms:

\begin{enumerate}
    \item \textbf{GitHub Actions} - Automatically on commit to the repository
    \item \textbf{Manual Execution} - Running the script directly from the command line
    \item \textbf{CI/CD Integration} - Integration with Jenkins, GitLab CI, or other systems
\end{enumerate}

\subsection{Command-Line Usage}
\label{Subsection::PipelineProgress::CommandLine}

\textbf{PowerShell:}
\begin{minted}{bash}
cd Book_SSW590_1
./build-pipeline.ps1 -OutputDir build -SkipZip:$false
\end{minted}

\textbf{Python:}
\begin{minted}{bash}
cd Book_SSW590_1
python build-pipeline.py --output-dir build --no-skip-zip
\end{minted}

\section{Build Artifacts}
\label{Section::PipelineProgress::Artifacts}

\subsection{Output Structure}
\label{Subsection::PipelineProgress::OutputStructure}

The pipeline generates the following output structure:

\begin{tcolorbox}[colback=gray!10, colframe=gray!50, title=Build Output Directory]
\small
\texttt{build/}\\
\texttt{├── main.pdf} \quad\quad\quad\quad\quad (Compiled LaTeX document)\\
\texttt{├── main.aux} \quad\quad\quad\quad\quad (LaTeX auxiliary files)\\
\texttt{├── main.toc, main.bbl} \quad\quad (Table of contents, bibliography)\\
\texttt{├── project/} \quad\quad\quad\quad\quad (Staged source files)\\
\texttt{│   ├── demo.js}\\
\texttt{│   ├── index.html}\\
\texttt{│   ├── style.css}\\
\texttt{│   ├── p5.js}\\
\texttt{│   ├── p5.sound.min.js}\\
\texttt{│   └── README.md}\\
\texttt{└── project.zip} \quad\quad\quad\quad (Packaged project archive)\\
\end{tcolorbox}

\subsection{File Sizes and Compression}
\label{Subsection::PipelineProgress::FileSizes}

The pipeline reports file sizes in the summary:
\begin{itemize}
    \item PDF size varies based on content and embedded images
    \item Project archive typically 500KB-2MB depending on library inclusion
    \item Compression ratio reflects the effectiveness of ZIP encoding
\end{itemize}

\section{Troubleshooting}
\label{Section::PipelineProgress::Troubleshooting}

\subsection{Common Issues}
\label{Subsection::PipelineProgress::CommonIssues}

\textbf{LaTeX Not Found:}
\begin{itemize}
    \item Windows: Install MiKTeX from \url{https://miktex.org/download}
    \item macOS: Install MacTeX via Homebrew: \texttt{brew install mactex}
    \item Linux: Install TeX Live: \texttt{sudo apt-get install texlive-full}
\end{itemize}

\textbf{PDF Not Generated:}
\begin{itemize}
    \item Check for LaTeX syntax errors in the document
    \item Verify all included files exist
    \item Review bibtex output for bibliography errors
    \item Check for missing LaTeX packages
\end{itemize}

\textbf{Archive Creation Failed:}
\begin{itemize}
    \item Verify project files exist and are readable
    \item Check disk space availability
    \item Ensure proper file permissions
\end{itemize}

\section{Summary}
\label{Section::PipelineProgress::Summary}

The pipeline implementation provides:
\begin{itemize}
    \item \textbf{Visibility} - Real-time progress reporting with color-coded status
    \item \textbf{Automation} - Fully automated build process from source to artifacts
    \item \textbf{Reliability} - Error handling and graceful degradation
    \item \textbf{Flexibility} - Multiple language implementations (PowerShell, Python)
    \item \textbf{Distribution} - Packaged artifacts ready for deployment
\end{itemize}

This comprehensive pipeline enables developers to track the complete build process from LaTeX compilation through project packaging in a single, unified interface.
