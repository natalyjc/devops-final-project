\chapter{Load Balancer \\
\small{\textit{-- Nataly Jimenez, Nicole Valdiviezo, Lakshya Vegiraju}}
\index{Load Balancer} 
\index{Chapter!Load Balancer}
\label{Chapter::Load Balancer and Virtual Host}}

\section{Overview}
The goal of this assignment is to create a simple load-balanced web application using Docker. Note that you will need to host this on Digital Ocean so that it can be tested by the TA. You will: Create two simple web servers serving different HTML content. Use Nginx as a reverse proxy to load balance between the servers. Deploy all components using Docker Compose

\subsection{Starting the Load Balancer}
The current project structure is this:

\begin{minted}{bash}
load-balanced-app/
    docker-compose.yml
    nginx/
        nginx.conf
    web1/
        index.html
    web2/
        index.html
\end{minted}

Web server content:
web1/index.html
\begin{minted}{html}
<h1>Hello from Web 1 </h1>
\end{minted}
web2/index.html
\begin{minted}{html}
<h1>Hello from Web 2 </h1>
\end{minted}

Nginx Load Balancer Configuration
nginx/nginx.conf
\begin{minted}{bash}
events {}
http {
    upstream backend {
        server web1:80;
        server web2:80;
    }
    server {
        listen 80;
        
        location / {
            proxy_pass http://backend;
            proxy_set_header Host $host;
            proxy_set_header X-Real-IP $remote_addr;
        }
    }
}
\end{minted}
Docker Compose File
docker-compose.yml
\begin{minted}{bash}
version: '3'
services:
    web1:
        image: nginx
        container_name: web1
        volumes:
        - ./web1:/usr/share/nginx/html:ro
    web2:
        image: nginx
        container_name: web2
        volumes:
        - ./web2:/usr/share/nginx/html:ro
    loadbalancer:
        image: nginx
        container_name: loadbalancer
        ports:
        - "8080:80"
        volumes:
        - ./nginx/nginx.conf:/etc/nginx/nginx.conf:ro
        depends_on:
        - web1
        - web 2
\end{minted}

The web servers where there tested locally using
\begin{minted}{bash}
docker-compose up --build
\end{minted}
When visiting http://localhost:8080 this is what appears:
\begin{figure}
    \centering
    \includegraphics[width=0.5\linewidth]{Book_SSW590_1/png/Screenshot 2025-11-12 174907.png}
\end{figure}

\begin{figure}
    \centering
    \includegraphics[width=0.5\linewidth]{Book_SSW590_1/png/Screenshot 2025-11-12 175547.png}
\end{figure}

\begin{figure}
    \centering
    \includegraphics[width=0.5\linewidth]{Book_SSW590_1/png/web12.png}
\end{figure}

\section{Virtual Hosting}

Virtual hosting allows a single server to host multiple domains or websites using the same physical machine. This is achieved by using the domain name in each incoming HTTP request to determine which configuration or content to serve. There are two main types of virtual hosting:

\begin{itemize}
    \item \textbf{Name-based virtual hosting:} Uses the \texttt{Host} header in the HTTP request to decide which website to serve. This is the most common method and does not require multiple IP addresses.
    \item \textbf{IP-based virtual hosting:} Each website is associated with a unique IP address.
\end{itemize}

This project focuses on name-based virtual hosting using Nginx.

\subsection{Nginx Configuration Example}
In this configuration, Nginx serves different content based on the hostname:

\begin{minted}{bash}
http {
    server {
        listen 80;
        server_name web1.domain.com;

        location / {
            proxy_pass http://web1:80;
            proxy_set_header Host $host;
            proxy_set_header X-Real-IP $remote_addr;
        }
    }

    server {
        listen 80;
        server_name web2.domain.com;

        location / {
            proxy_pass http://web2:80;
            proxy_set_header Host $host;
            proxy_set_header X-Real-IP $remote_addr;
        }
    }
}
\end{minted}

\noindent
In this setup:
\begin{itemize}
    \item Requests to \texttt{web1.domain.com} are routed to the \texttt{web1} container.
    \item Requests to \texttt{web2.domain.com} are routed to the \texttt{web2} container.
\end{itemize}

\subsection{Local Testing}
To test locally without owning real domains, you can simulate domain routing by editing your \texttt{/etc/hosts} file:

\begin{minted}{bash}
127.0.0.1 web1.domain.com
127.0.0.1 web2.domain.com
\end{minted}

After this change, visiting \url{http://web1.domain.com} or \url{http://web2.domain.com} in a browser will correctly route through the Nginx proxy.

\subsection{Docker-Based Virtual Hosting}
Using Docker Compose, each service can be uniquely named and accessed via Nginx virtual hosts:

\begin{minted}{bash}
version: '3'
services:
    web1:
        image: nginx
        volumes:
            - ./web1:/usr/share/nginx/html

    web2:
        image: nginx
        volumes:
            - ./web2:/usr/share/nginx/html

    proxy:
        image: nginx
        ports:
            - "80:80"
        volumes:
            - ./nginx/nginx.conf:/etc/nginx/nginx.conf:ro
\end{minted}

This configuration enables Nginx to route requests based on the domain name to the correct backend container.

\subsection{Conclusion}
Virtual hosting is a flexible and scalable method for running multiple websites from a single server. By combining Docker and Nginx, developers can easily configure isolated web environments for different domains or projects, simplifying deployment and management.

\section{Deliverables}
\begin{enumerate}
    \item \textbf{Access Links:}
    \begin{itemize}
        \item \href{http://web1.domain.com:8082}{http://web1.domain.com:8082}
        \item \href{http://web2.domain.com:8082}{http://web2.domain.com:8082}
    \end{itemize}

    \item \textbf{Screenshots:}
    \begin{figure}[H]
        \centering
        \includegraphics[width=0.8\textwidth]{Book_SSW590_1/png/hellofromweb1.png}
        \caption{Web1 page loaded successfully.}
    \end{figure}

    \begin{figure}[H]
        \centering
        \includegraphics[width=0.8\textwidth]{Book_SSW590_1/png/hellofromweb2.png}
        \caption{Web2 page loaded successfully.}
    \end{figure}
\end{enumerate}