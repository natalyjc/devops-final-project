\chapter{Jenkins With Pytest \\
\small{\textit{-- Nicole Valdiviezo}}
\index{Jenkins With Pytest} 
\index{Chapter!Jenkins With Pytest}
\label{Chapter::Jenkins With Pytest}}

\section{Introduction}
This report documents the steps taken to set up and configure a Continuous Integration and Continuous Deployment (CI/CD) pipeline using \textbf{Jenkins} running inside Docker. The pipeline integrates with a simple Python project tested using \textbf{Pytest} and utilizes Git for source control. The goal of this exercise was to gain hands-on experience in automating testing and reporting through Jenkins pipelines.

\section{Part 1: Jenkins Setup in Docker}
\subsection{Docker Compose Configuration}
The first step was to define a Docker Compose file to run Jenkins as a containerized service. Below is the content of the \texttt{docker-compose.yml} file used:

\begin{minted}{bash}
version: '3.8'
services:
  jenkins:
    image: jenkins/jenkins:lts
    container_name: jenkins
    ports:
      - "8080:8080"
      - "50000:50000"
    volumes:
      - jenkins_home:/var/jenkins_home
      - /var/run/docker.sock:/var/run/docker.sock

volumes:
  jenkins_home:
\end{minted}

After creating the file, Jenkins was started using the following command:
\begin{minted}{bash}
docker compose up -d
\end{minted}

\section{Accessing Jenkins}
Once Jenkins was running, it was accessed through a web browser at:
\begin{minted}{bash}
http://localhost:8080
\end{minted}

To unlock Jenkins, the initial admin password was retrieved by running:
\begin{minted}{bash}
docker exec -it jenkins cat /var/jenkins_home/secrets/initialAdminPassword
\end{minted}

After logging in, the \textbf{suggested plugins} were installed, and an admin user was created.

\section{Part 2: Python + Pytest Project}
\section{Directory Structure}
The following directory structure was used for the Python project:

\begin{minted}{bash}
jenkins-python-pytest-demo/
│
├── Jenkinsfile
├── requirements.txt
└── tests/
    └── test_sample.py
\end{minted}

\section{Python Files}
\textbf{requirements.txt}
\begin{minted}{bash}
pytest
\end{minted}

\begin{minted}{python}
def test_addition():
    assert 1 + 1 == 2

def test_subtraction():
    assert 5 - 2 == 3

def test_failure_example():
    assert 2 * 2 == 5  # This will fail intentionally
\end{minted}

\section{Part 3: Jenkins Pipeline Configuration}
\subsection{Jenkinsfile Content}
The pipeline was defined using a declarative Jenkinsfile, as shown below:

\begin{minted}{bash}
pipeline {
    agent any
    stages {
        stage('Install dependencies') {
            steps {
                sh 'python3 -m venv venv'
                sh './venv/bin/pip install -r requirements.txt'
            }
        }
        stage('Run tests') {
            steps {
                sh './venv/bin/pytest --junitxml=report.xml'
            }
        }
        stage('Publish Report') {
            steps {
                junit 'report.xml'
            }
        }
    }
}
\end{minted}

This file instructs Jenkins to:
\begin{itemize}
    \item Create a Python virtual environment.
    \item Install dependencies from \texttt{requirements.txt}.
    \item Run Pytest and generate a JUnit XML report.
    \item Publish the test report within Jenkins.
\end{itemize}


\section{Part 4: Git Integration}
\subsection{Local Git Repository Setup}
The project was initialized as a Git repository using:
\begin{minted}{bash}
git init
git add .
git commit -m "Initial commit"
\end{minted}

Optionally, it was pushed to GitHub:
\begin{minted}{bash}
git remote add origin https://github.com/yourusername/jenkins-python-pytest-demo.git
git branch -M main
git push -u origin main
\end{minted}

\section{Connecting Jenkins to GitHub}
In Jenkins:
\begin{enumerate}
    \item Installed the \textbf{Git Plugin}.
    \item Created a new \textbf{Pipeline job}.
    \item Selected \textbf{Pipeline script from SCM}.
    \item Chose \textbf{Git} as the SCM.
    \item Entered the GitHub repository URL.
    \item Set the branch to \texttt{main}.
\end{enumerate}


\section{Part 5: Viewing Test Reports}
After building the Jenkins job:
\begin{itemize}
    \item The \textbf{Console Output} was reviewed to confirm successful installation and testing.
    \item The \textbf{Test Result} tab displayed pass/fail outcomes.
    \item The \textbf{Stage View} provided a visual overview of pipeline stages.
\end{itemize}


\section{Conclusion}
Through this assignment, a complete Jenkins-based CI/CD pipeline was successfully set up and tested. The pipeline automates dependency installation, test execution, and report generation, demonstrating the power of Jenkins for software automation and continuous integration.

\section{Deliverables}

\begin{itemize}
    \item Jenkins at \url{http://localhost:8080}
    \item GitHub: \url{https://github.com/natalyjc/devops.git}
\end{itemize}


\begin{figure}[ht]
    \centering
    \includegraphics[width=0.6\linewidth]{Book_SSW590_1/png/Screenshot (100).png}
    \caption{Jenkins Screenshot}
    \label{Jenkins Screenshot}
\end{figure}

\begin{figure}[ht]
    \centering
    \includegraphics[width=0.6\linewidth]{Book_SSW590_1/png/Screenshot (101).png}
    \caption{Jenkins Screenshot}
    \label{Jenkins Screenshot}
\end{figure}

\begin{figure}[ht]
    \centering
    \includegraphics[width=0.6\linewidth]{Book_SSW590_1/png/Screenshot (102).png}
    \caption{Jenkins Screenshot}
    \label{Jenkins Screenshot}
\end{figure}